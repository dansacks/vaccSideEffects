%===============================================================================
% Exhibits Document
%
% Compiles all tables and figures from the VaccSideEffects project.
%
% To compile:
%   pdflatex exhibits.tex
%
% Created by Dan + Claude Code
%===============================================================================

\documentclass[11pt]{article}

% Page setup
\usepackage[margin=1in]{geometry}
\usepackage{pdflscape}

% Tables
\usepackage{booktabs}
\usepackage{tabularx}
\usepackage{multirow}
\usepackage{array}
\usepackage{longtable}

% Figures
\usepackage{graphicx}
\usepackage{float}

% Formatting
\usepackage{setspace}
\onehalfspacing

% Caption formatting
\usepackage{caption}
\captionsetup{labelfont=bf, justification=raggedright, singlelinecheck=false}

% Title
\title{VaccSideEffects Study Exhibits}
\author{}
\date{}

\begin{document}

\maketitle
\tableofcontents
\newpage

%===============================================================================
\section{Sample Counts}
%===============================================================================

\begin{table}[H]
\caption{Sample Size by Survey Stage}
\label{tab:counts}
\centering
\begin{tabular}{llr}
\toprule
Survey & Condition & Count \\
\midrule
\multicolumn{3}{l}{\textit{Prescreen}} \\
& Started (non-preview) & 8,198 \\
& Consented & 8,093 \\
& Passed attention check & 8,052 \\
& First attempt only & 8,003 \\
& Hesitant (final prescreen sample) & 4,525 \\
& Matches to demographics & 4,525 \\
\addlinespace
\multicolumn{3}{l}{\textit{Main}} \\
& Started (non-preview) & 3,651 \\
& Linked to prescreen final & 3,637 \\
& Consented & 3,635 \\
& Passed attention check & 3,619 \\
& Non-missing delta (valid posteriors) & 3,547 \\
& First attempt only (final main sample) & 3,526 \\
& Matches to demographics & 3,526 \\
\addlinespace
\multicolumn{3}{l}{\textit{Followup}} \\
& Started (non-preview) & 3,210 \\
& Linked to main final & 3,178 \\
& Passed attention check & 3,134 \\
& Has non-missing outcome & 3,074 \\
& First attempt only (final followup sample) & 3,046 \\
& Matches to demographics & 3,046 \\
\bottomrule
\end{tabular}
\end{table}


%===============================================================================
\section{Balance Tables}
%===============================================================================

%-------------------------------------------------------------------------------
\subsection{Main Balance Table}
%-------------------------------------------------------------------------------

\begin{table}[H]
\caption{Covariate Balance Across Treatment Arms}
\label{tab:balance_main}
\centering
\begin{tabular}{lccccc}
\toprule
Variable & Control & Industry & Academic & Personal & P-value \\
\midrule
Prior: SE likely with vaccine & 0.551 & 0.541 & 0.533 & 0.529 & 0.794 \\
Do not intend to vaccinate & 0.655 & 0.666 & 0.674 & 0.662 & 0.870 \\
Previously had flu vaccine & 0.600 & 0.624 & 0.560 & 0.592 & 0.055 \\
Previously had COVID vaccine & 0.627 & 0.642 & 0.625 & 0.632 & 0.876 \\
Severe flu vaccine reaction & 0.108 & 0.093 & 0.089 & 0.105 & 0.703 \\
Severe COVID vaccine reaction & 0.170 & 0.144 & 0.135 & 0.168 & 0.256 \\
Has health condition & 0.169 & 0.218 & 0.168 & 0.182 & 0.033 \\
Age 18--34 & 0.351 & 0.343 & 0.333 & 0.322 & 0.588 \\
Age 35--49 & 0.400 & 0.386 & 0.420 & 0.389 & 0.444 \\
White & 0.756 & 0.750 & 0.745 & 0.755 & 0.944 \\
Hispanic & 0.080 & 0.095 & 0.099 & 0.091 & 0.547 \\
Income under 50k & 0.350 & 0.399 & 0.382 & 0.359 & 0.146 \\
Trust government & 0.471 & 0.451 & 0.453 & 0.447 & 0.773 \\
Follow doctor advice & 0.568 & 0.550 & 0.548 & 0.566 & 0.755 \\
College degree & 0.535 & 0.505 & 0.495 & 0.519 & 0.363 \\
Joint test & \multicolumn{4}{c}{$\chi^2(45)=42.508$} & 0.578 \\
N & 885 & 885 & 882 & 886

\\
\bottomrule
\end{tabular}
\begin{minipage}{\textwidth}
\vspace{0.5em}
\footnotesize
\textit{Notes:} Cell entries are group means. P-values are from F-tests of joint equality across treatment arms with robust standard errors. Joint test is a chi-squared test of all coefficients jointly equal to zero using seemingly unrelated regression.
\end{minipage}
\end{table}

%-------------------------------------------------------------------------------
\subsection{Balance by Domain}
%-------------------------------------------------------------------------------

\subsubsection{Prior Beliefs}

\begin{table}[H]
\caption{Balance Table: Prior Beliefs}
\label{tab:balance_prior}
\centering
\begin{tabular}{lccccc}
\toprule
Variable & Control & Industry & Academic & Personal & P-value \\
\midrule
\input{tables/balance_prior_beliefs.tex}
\\
\bottomrule
\end{tabular}
\begin{minipage}{\textwidth}
\vspace{0.5em}
\footnotesize
\textit{Notes:} See Table \ref{tab:balance_main} for details.
\end{minipage}
\end{table}

\subsubsection{Vaccination Intent}

\begin{table}[H]
\caption{Balance Table: Vaccination Intent}
\label{tab:balance_intent}
\centering
\begin{tabular}{lccccc}
\toprule
Variable & Control & Industry & Academic & Personal & P-value \\
\midrule
Intent: No, do not intend (vs. may or may not) & 0.655 & 0.666 & 0.674 & 0.662 & 0.870 \\
\addlinespace
N & 885 & 885 & 882 & 886 & \\

\\
\bottomrule
\end{tabular}
\begin{minipage}{\textwidth}
\vspace{0.5em}
\footnotesize
\textit{Notes:} See Table \ref{tab:balance_main} for details.
\end{minipage}
\end{table}

\subsubsection{Vaccine Experience}

\begin{table}[H]
\caption{Balance Table: Vaccine Experience}
\label{tab:balance_vacc_exp}
\centering
\begin{tabular}{lccccc}
\toprule
Variable & Control & Industry & Academic & Personal & P-value \\
\midrule
Had prior COVID vaccine & 0.627 & 0.642 & 0.625 & 0.632 & 0.876 \\
Had prior flu vaccine & 0.600 & 0.624 & 0.560 & 0.592 & 0.055 \\
COVID reaction: None/don't remember & 0.416 & 0.399 & 0.416 & 0.438 & 0.616 \\
COVID reaction: Mild & 0.414 & 0.457 & 0.449 & 0.394 & 0.104 \\
COVID reaction: Severe & 0.170 & 0.144 & 0.135 & 0.168 & 0.256 \\
Flu reaction: None/don't remember & 0.546 & 0.511 & 0.489 & 0.525 & 0.310 \\
Flu reaction: Mild & 0.346 & 0.396 & 0.422 & 0.370 & 0.072 \\
Flu reaction: Severe & 0.108 & 0.093 & 0.089 & 0.105 & 0.703 \\
\addlinespace
Joint test & \multicolumn{4}{c}{$\chi^2(18)=22.50$} & 0.211 \\
\addlinespace
N & 885 & 885 & 882 & 886 & \\

\\
\bottomrule
\end{tabular}
\begin{minipage}{\textwidth}
\vspace{0.5em}
\footnotesize
\textit{Notes:} Reaction variables are conditional on having received the respective vaccine. See Table \ref{tab:balance_main} for details.
\end{minipage}
\end{table}


\newpage
\subsubsection{Demographics}

\begin{landscape}
\begin{table}[H]
\caption{Balance Table: Demographics}
\label{tab:balance_demographics}
\centering
\small
\begin{tabular}{lccccc}
\toprule
Variable & Control & Industry & Academic & Personal & P-value \\
\midrule
Age 18--34 & 0.351 & 0.343 & 0.333 & 0.322 & 0.588 \\
Age 35--49 & 0.400 & 0.386 & 0.420 & 0.389 & 0.444 \\
Age 50--64 & 0.213 & 0.221 & 0.221 & 0.227 & 0.918 \\
Age 65+ & 0.034 & 0.046 & 0.025 & 0.061 & 0.001 \\
Female & 0.542 & 0.558 & 0.566 & 0.547 & 0.745 \\
Gender: Other & 0.007 & 0.009 & 0.008 & 0.012 & 0.674 \\
Education: HS or less & 0.152 & 0.138 & 0.153 & 0.166 & 0.442 \\
Education: Some college & 0.311 & 0.354 & 0.351 & 0.315 & 0.097 \\
Education: 4-year degree & 0.375 & 0.350 & 0.371 & 0.365 & 0.688 \\
Education: Graduate degree & 0.158 & 0.152 & 0.123 & 0.153 & 0.123 \\
Income: Under \$25k & 0.138 & 0.154 & 0.142 & 0.121 & 0.230 \\
Income: \$25--50k & 0.204 & 0.231 & 0.232 & 0.230 & 0.405 \\
Income: \$50--75k & 0.213 & 0.195 & 0.210 & 0.225 & 0.479 \\
Income: \$75--100k & 0.169 & 0.148 & 0.155 & 0.161 & 0.668 \\
Income: Over \$100k & 0.253 & 0.236 & 0.243 & 0.242 & 0.880 \\
White & 0.756 & 0.750 & 0.745 & 0.755 & 0.944 \\
Black & 0.125 & 0.136 & 0.148 & 0.135 & 0.565 \\
Asian & 0.006 & 0.012 & 0.007 & 0.003 & 0.184 \\
American Indian/Alaska Native & 0.075 & 0.053 & 0.046 & 0.061 & 0.066 \\
Race: Other & 0.005 & 0.000 & 0.000 & 0.002 & 0.050 \\
Hispanic & 0.080 & 0.095 & 0.099 & 0.091 & 0.547 \\
Very liberal & 0.083 & 0.111 & 0.083 & 0.098 & 0.142 \\
Liberal & 0.171 & 0.170 & 0.191 & 0.172 & 0.605 \\
Slightly liberal & 0.127 & 0.124 & 0.112 & 0.124 & 0.746 \\
Moderate & 0.236 & 0.210 & 0.241 & 0.233 & 0.413 \\
Slightly conservative & 0.109 & 0.118 & 0.113 & 0.111 & 0.945 \\
Conservative & 0.189 & 0.184 & 0.169 & 0.183 & 0.693 \\
Very conservative & 0.077 & 0.069 & 0.081 & 0.068 & 0.672 \\
\addlinespace
Joint test & \multicolumn{4}{c}{$\chi^2(83)=94.54$} & 0.182 \\
\addlinespace
N & 885 & 885 & 882 & 886 &

\\
\bottomrule
\end{tabular}
\begin{minipage}{\linewidth}
\vspace{0.5em}
\footnotesize
\textit{Notes:} See Table \ref{tab:balance_main} for details.
\end{minipage}
\end{table}
\end{landscape}


\subsubsection{Trust and Health Conditions}

\begin{table}[H]
\caption{Balance Table: Trust and Health Conditions}
\label{tab:balance_trust}
\centering
\begin{tabular}{lccccc}
\toprule
Variable & Control & Industry & Academic & Personal & P-value \\
\midrule
Trust govt: Strongly disagree & 0.119 & 0.129 & 0.122 & 0.128 & 0.911 \\
Trust govt: Somewhat disagree & 0.177 & 0.189 & 0.190 & 0.153 & 0.120 \\
Trust govt: Neither & 0.234 & 0.231 & 0.235 & 0.272 & 0.155 \\
Trust govt: Somewhat agree & 0.358 & 0.345 & 0.344 & 0.349 & 0.925 \\
Trust govt: Strongly agree & 0.112 & 0.106 & 0.109 & 0.098 & 0.796 \\
No health conditions & 0.831 & 0.782 & 0.832 & 0.818 & 0.033 \\
Has asthma & 0.111 & 0.129 & 0.092 & 0.115 & 0.095 \\
Has lung disease & 0.008 & 0.008 & 0.009 & 0.015 & 0.536 \\
Has heart disease & 0.014 & 0.038 & 0.023 & 0.019 & 0.015 \\
Has diabetes & 0.042 & 0.060 & 0.056 & 0.045 & 0.252 \\
Has kidney disease & 0.008 & 0.018 & 0.008 & 0.010 & 0.235 \\
Health conditions: rather not say & 0.005 & 0.001 & 0.006 & 0.005 & 0.206 \\
\addlinespace
Joint test & \multicolumn{4}{c}{$\chi^2(33)=43.23$} & 0.110 \\
\addlinespace
N & 885 & 885 & 882 & 886 & \\

\\
\bottomrule
\end{tabular}
\begin{minipage}{\textwidth}
\vspace{0.5em}
\footnotesize
\textit{Notes:} See Table \ref{tab:balance_main} for details.
\end{minipage}
\end{table}

%-------------------------------------------------------------------------------
\subsection{Omnibus Balance Test}
%-------------------------------------------------------------------------------

\begin{table}[H]
\caption{Omnibus Balance Test Across All Domains}
\label{tab:balance_omnibus}
\centering
\begin{tabular}{lccccc}
\toprule
Test & \multicolumn{4}{c}{Statistic} & P-value \\
\midrule
Omnibus (all domains) & \multicolumn{4}{c}{$\chi^2(173)=204.71$} & 0.0499 \\
Omnibus (excl. demographics) & \multicolumn{4}{c}{$\chi^2(90)=105.05$} & 0.1327

\\
\bottomrule
\end{tabular}
\begin{minipage}{\textwidth}
\vspace{0.5em}
\footnotesize
\textit{Notes:} Joint chi-squared test of all balance variables across all domains. Uses seemingly unrelated regression with robust standard errors.
\end{minipage}
\end{table}

%===============================================================================
\section{Treatment Effects}
%===============================================================================

\begin{table}[H]
\caption{Treatment Effects on Primary Outcomes}
\label{tab:treatment_effects}
\centering
\begin{tabular}{lccccc}
\toprule
& \multicolumn{5}{c}{Outcome} \\
\cmidrule(lr){2-6}
& Post-Trial & Posterior & Vaccination & Link & Follow-up \\
& SE Estimate & Difference & Intent & Click & Vaccinated \\
\midrule

                    belief SAE    belief     will       link            follow-up 
                    in trial      delta     or may      click          vaccinated (=1 if vaccinated in last month, or say you previously vaccinated)
Industry        &    -8.181 &    -3.576 &    -0.001 &    -0.002 &    -0.014 \\
                & (  0.848) & (  1.017) & (  0.012) & (  0.006) & (  0.014) \\
Academic        &    -4.385 &    -2.819 &     0.031 &    -0.004 &     0.021 \\
                & (  0.875) & (  1.011) & (  0.013) & (  0.006) & (  0.015) \\
Personal        &    -4.818 &     0.347 &     0.024 &    -0.002 &     0.016 \\
                & (  0.879) & (  1.018) & (  0.013) & (  0.006) & (  0.014) \\
Control mean    &    20.226 &    15.437 &     0.069 &     0.015 &     0.084 \\
N               &     3,499 &     3,499 &     3,499 &     3,499 &     2,974 \\
R-squared       &     0.164 &     0.371 &     0.136 &     0.031 &     0.116 \\

\\
\bottomrule
\end{tabular}
\begin{minipage}{\textwidth}
\vspace{0.5em}
\footnotesize
\textit{Notes:} OLS regression coefficients with robust standard errors in parentheses. Control arm is the omitted category. All specifications include controls for prior beliefs, vaccination intent, vaccine experience, demographics, trust in government, and health conditions. Post-Trial SE Estimate is the respondent's estimate of side effect rate (0-100). Posterior Difference is posterior belief with vaccine minus posterior belief without vaccine. Vaccination Intent is 1 if respondent intends to or already got the flu vaccine. Link Click is 1 if respondent clicked any information link. Follow-up Vaccinated is 1 if respondent reported getting flu vaccine or had already been vaccinated.
\end{minipage}
\end{table}

%===============================================================================
\section{Persistence of Information}
%===============================================================================

%-------------------------------------------------------------------------------
\subsection{Differential Attrition}
%-------------------------------------------------------------------------------

\begin{table}[H]
\caption{Differential Attrition by Treatment Arm}
\label{tab:persistence_attrition}
\centering
\begin{tabular}{lccc}
\toprule
& In & Recall & SE \\
& Followup & Sample & Sample \\
\midrule
Industry        &     0.018 &     0.020 &     0.028 \\
                & (  0.016) & (  0.016) & (  0.017) \\
Academic        &     0.007 &     0.008 &     0.008 \\
                & (  0.017) & (  0.017) & (  0.018) \\
Personal        &     0.002 &     0.003 &     0.007 \\
                & (  0.017) & (  0.017) & (  0.017) \\
Joint p-value   &     0.674 &     0.590 &     0.388 \\
Control mean    &     0.858 &     0.854 &     0.835 \\
N               &     3,516 &     3,516 &     3,516

\\
\bottomrule
\end{tabular}
\begin{minipage}{\textwidth}
\vspace{0.5em}
\footnotesize
\textit{Notes:} OLS regression coefficients with robust standard errors in parentheses. Control arm is the omitted category. All specifications include controls. Joint p-value is from an F-test of joint significance of treatment arm indicators.
\end{minipage}
\end{table}

%-------------------------------------------------------------------------------
\subsection{Recall of Study Participation}
%-------------------------------------------------------------------------------

\begin{table}[H]
\caption{Recall of Study Participation and Information Sources}
\label{tab:persistence_recall}
\centering
\begin{tabular}{lcccc}
\toprule
& Recall & Recall & Recall & Recall \\
& Study & Manufacturer & University & GAVI \\
\midrule
Industry        &    -0.008 &     0.043 &     0.023 &    -0.013 \\
                & (  0.022) & (  0.027) & (  0.028) & (  0.018) \\
Academic        &     0.018 &     0.072 &     0.072 &     0.003 \\
                & (  0.023) & (  0.028) & (  0.028) & (  0.018) \\
Personal        &     0.017 &     0.048 &     0.050 &    -0.002 \\
                & (  0.022) & (  0.027) & (  0.028) & (  0.018) \\
Control mean    &     0.247 &     0.349 &     0.480 &     0.109 \\
N               &     2,995 &     2,515 &     2,515 &     2,515

\\
\bottomrule
\end{tabular}
\begin{minipage}{\textwidth}
\vspace{0.5em}
\footnotesize
\textit{Notes:} OLS regression coefficients with robust standard errors in parentheses. Control arm is the omitted category. All specifications include controls. Sample restricted to followup respondents with valid recall responses.
\end{minipage}
\end{table}

%-------------------------------------------------------------------------------
\subsection{Memory of Adverse Event Rates}
%-------------------------------------------------------------------------------

\begin{table}[H]
\caption{Memory of Clinical Trial Adverse Event Rates}
\label{tab:persistence_adverse}
\centering
\begin{tabular}{lccccc}
\toprule
& Guess & Placebo & Guess & Vaccine & Guess \\
& Placebo & Correct & Vaccine & Correct & Delta \\
\midrule
Industry &    -0.052 &     0.001 &    -2.698 &     0.002 &    -2.646 \\
                & (  0.848) & (  0.022) & (  0.812) & (  0.017) & (  0.956) \\
Academic &     0.795 &     0.005 &    -0.716 &     0.002 &    -1.511 \\
                & (  0.873) & (  0.022) & (  0.880) & (  0.017) & (  0.987) \\
Personal &     1.019 &    -0.020 &    -1.308 &     0.035 &    -2.327 \\
                & (  0.864) & (  0.021) & (  0.860) & (  0.018) & (  1.006) \\
Control mean    &    15.545 &     0.226 &    15.221 &     0.115 &    -0.323 \\
N               & 2,988 & 2,988 & 2,988 & 2,988 & 2,988

\\
\bottomrule
\end{tabular}
\begin{minipage}{\textwidth}
\vspace{0.5em}
\footnotesize
\textit{Notes:} OLS regression coefficients with robust standard errors in parentheses. Control arm is the omitted category. All specifications include controls. Sample restricted to followup respondents with valid adverse event rate guesses.
\end{minipage}
\end{table}

%===============================================================================
\section{Belief Distribution Figures}
%===============================================================================

%-------------------------------------------------------------------------------
\subsection{Pooled Belief Distribution}
%-------------------------------------------------------------------------------

\begin{figure}[H]
\centering
\includegraphics[width=0.9\textwidth]{figures/beliefs_pooled.png}
\caption{Cumulative Distribution of Posterior Believed Side Effect Rate}
\label{fig:beliefs_pooled}
\begin{minipage}{\textwidth}
\vspace{0.5em}
\footnotesize
\textit{Notes:} Shows the CDF of $\Delta$ = posterior belief with vaccine minus posterior belief without vaccine. Clinical trial estimates suggest the true effect is between -1.7 and 1.0 percentage points.
\end{minipage}
\end{figure}

%-------------------------------------------------------------------------------
\subsection{Beliefs by Vaccine Experience}
%-------------------------------------------------------------------------------

\begin{figure}[H]
\centering
\includegraphics[width=0.8\textwidth]{figures/delta_by_vacc_reaction.png}
\caption{Mean Posterior Belief by Vaccine Reaction Experience}
\label{fig:delta_vacc_reaction}
\begin{minipage}{\textwidth}
\vspace{0.5em}
\footnotesize
\textit{Notes:} Bars show mean $\Delta$ (posterior belief with vaccine minus without) by self-reported reaction to previous flu and COVID vaccines. Error bars are 95\% confidence intervals.
\end{minipage}
\end{figure}

%-------------------------------------------------------------------------------
\subsection{Beliefs by Trust}
%-------------------------------------------------------------------------------

\begin{figure}[H]
\centering
\includegraphics[width=0.8\textwidth]{figures/delta_by_trust.png}
\caption{Mean Posterior Belief by Trust in Government and Doctor Adherence}
\label{fig:delta_trust}
\begin{minipage}{\textwidth}
\vspace{0.5em}
\footnotesize
\textit{Notes:} Bars show mean $\Delta$ by self-reported trust in government vaccine recommendations and likelihood of following doctor's vaccine advice. Error bars are 95\% confidence intervals.
\end{minipage}
\end{figure}

%-------------------------------------------------------------------------------
\subsection{Belief Distribution by Treatment Arm}
%-------------------------------------------------------------------------------

\begin{figure}[H]
\centering
\includegraphics[width=\textwidth]{figures/belief_cdf_by_arm.png}
\caption{Belief CDFs by Treatment Arm Relative to Control}
\label{fig:beliefs_by_arm}
\begin{minipage}{\textwidth}
\vspace{0.5em}
\footnotesize
\textit{Notes:} Each panel shows the CDF of $\Delta$ for one treatment arm (colored) compared to the control arm (gray). Treatment arms received information about flu vaccine side effects from different sources: Industry (pharmaceutical company), Academic (university researchers), or Personal (personally relevant framing).
\end{minipage}
\end{figure}

\end{document}
