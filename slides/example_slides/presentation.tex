% ********************************************************************************************
% preamble
% ********************************************************************************************

% *****************************************************************
% define document class and load packages
% *****************************************************************

\documentclass[11.5pt,svgnames,x11names,dvipsnames,aspectratio=169]{beamer}
\usepackage[T1]{fontenc}
\usepackage[latin9]{inputenc}
%\usepackage[utf8x]{inputenc}
\usepackage{lastpage}
\usepackage{hyperref}
\usepackage{amsthm}
\usepackage{amssymb}
\usepackage{amsmath}
\usepackage{mathtools}
\usepackage{graphicx}
\usepackage{floatrow}
\usepackage{setspace}
\usepackage{bbm}
\usepackage{bm}
\usepackage{booktabs}
\usepackage{pdflscape}
\usepackage{dcolumn}
\usepackage{threeparttable}
\usepackage{siunitx}
\usepackage{caption}
\usepackage{array}
\usepackage{subfig}
\usepackage[countmax]{subfloat}
\usepackage{ragged2e}
\usepackage{calc}
\usepackage[normalem]{ulem}
\usepackage{pgfplots}
\usepackage{tikz}
\usetikzlibrary{shapes,snakes,arrows}
\usepackage{lastpage}
\usepackage{textpos}
\usepackage{hyperref}
\usepackage{cases}
\usepackage{datetime}
\usepackage{collcell}
\usepackage[default]{lato}
\usepackage[thicklines]{cancel}
\usepackage{newunicodechar}
\usepackage{pifont}
\usepackage{amssymb}
\newcommand{\cmark}{\ding{51}}
\newcommand{\xmark}{\ding{55}}

\makeatletter
  \def\Hy@PageAnchorSlidesPlain{}%
  \def\Hy@PageAnchorSlide{}%
\makeatother

%tikz stuff
\usepackage{tikz}
\usepackage{pgfplots}
\pgfplotsset{compat=newest}
\usetikzlibrary{arrows}
\usetikzlibrary{decorations.markings}
\usetikzlibrary{snakes}
\usetikzlibrary{tikzmark}
\usetikzlibrary{external}
\tikzexternalize[prefix=figures/tikz/, optimize = false]

% *****************************************************************
% some style settings
% *****************************************************************

% general
\setbeamercovered{invisible}

% margins
\setbeamersize{text margin left = \leftmargini, text margin right=\leftmargini}
\setlength{\leftmargini}{10pt}

% fonts for titles
\setbeamerfont{title}{series=\bfseries}
\setbeamerfont{frametitle}{series=\bfseries}

%define some colors
\definecolor{dknavy}{RGB}{30, 45, 83}
\definecolor{edkblue}{RGB}{62, 100, 125}
\definecolor{ebblue}{RGB}{0, 139, 188}
\definecolor{orangered}{RGB}{255, 69, 0}
\definecolor{gs12}{RGB}{192, 192, 192}
\definecolor{gs15}{RGB}{240, 240, 240}

% title colors
\setbeamercolor{title}{fg = dknavy, bg = white}
\setbeamercolor{frametitle}{fg = dknavy, bg = white}
\setbeamercolor{title in head/foot}{fg = dknavy, bg = gs15}

% itemize bullet colors
\setbeamercolor{itemize item}{fg = orangered}
\setbeamercolor{itemize subitem}{fg = edkblue}
\setbeamercolor{itemize subsubitem}{fg = edkblue}

% enumerate colors
\setbeamertemplate{enumerate items}[default]
\setbeamercolor{enumerate item}{fg=edkblue}
\setbeamercolor{enumerate subitem}{fg=title.fg}
\setbeamertemplate{enumerate item}{\textbf{\arabic{enumi}.}}

% text block colors
\setbeamercolor{block title}{bg = title.bg, fg = title.fg}
\setbeamercolor{block body}{bg = title.bg}

% alerted text colors
\setbeamercolor{alerted text}{fg=orangered}

% button colors
\setbeamercolor{button}{bg=gs15,fg=title.fg}

% itemize bullet shapes
\setbeamertemplate{itemize items}[circle]
\setbeamertemplate{itemize subitem}[circle]
\setbeamertemplate{itemize subsubitem}{--}

% header/footer
\setbeamertemplate{headline}{}
\setbeamertemplate{footline}[frame number]{}
\setbeamertemplate{navigation symbols}{}

%format and settings for section title pages
\setbeamertemplate{section page}
{
	\setbeamercolor{background canvas}{bg=edkblue}
    \begin{frame}[plain,noframenumbering]    
    \vfill
    \begin{center}
    \usebeamerfont{section title}\textcolor{white}{\textbf{\insertsection}}
    \end{center}
    \vfill
    \end{frame}
}

% theorem options
\makeatletter
\def\th@mystyle{%
    \normalfont % body font
    \setbeamercolor{block title example}{bg=edkblue!80,fg=white}
    \setbeamercolor{block body example}{bg=gs15,fg=black}
    \def\inserttheoremblockenv{exampleblock}
  }
\makeatother
\theoremstyle{mystyle}

\newtheorem{proposition}[theorem]{\textbf{Proposition}}
\newtheorem{cor}[theorem]{\textbf{Corollary}}

% *****************************************************************
% table and figure options 
% *****************************************************************

\DeclareCaptionLabelSeparator*{emdash}{.---}

\captionsetup[figure]{
	position = top,
	format = plain,
	labelsep = emdash,
	labelfont = sc,
	font = footnotesize,
	justification = centering
}

\captionsetup[subfigure]{
	position = top,
	format = plain,
	labelfont = normalfont,
	justification = centering
}

\captionsetup[table]{
	format = plain,
	labelsep = newline,
	labelfont = sc,
	font = {sc, small},
	skip = 0pt,
	justification = centering
}

\floatsetup[table]{style=plaintop, midcode = captionskip, captionskip = 0pt}
\floatsetup[figure]{midcode = captionskip, captionskip = 5pt}

% siunitx package options
\sisetup{
	detect-mode,
	tight-spacing = true,
	group-digits = false ,
	input-decimal-markers = {.},
	input-symbols = {( ) [ ] - + * ,}
}

% *****************************************************************
% things related to importing tables from Stata's "estout" command
% *****************************************************************

% define a new input command so that we can still flatten the document
\makeatletter
\let\estinput=\@@input
%\let\estinput=\input

% align significance stars immediately next to estimate
\newcommand{\sym}[1]{\rlap{#1}}

% wrapper for tables produced by estout
\newcommand{\estauto}[3]{
	\vspace{.75ex}{
		\begin{tabular}{l*{#2}{#3}}
			\toprule
			\estinput{#1} 
			\bottomrule 
			\addlinespace[.75ex]
		\end{tabular}
	}
}		

% Allow line breaks with \\ in specialcells
\newcommand{\specialcell}[2][c]{
	\begin{tabular}[#1]{@{}c@{}}#2\end{tabular}
}

% ********************************************************************************************
% main presentation body
% ********************************************************************************************

\begin{document}

% *****************************************************************
% title slide
% *****************************************************************

\title[]{Real Effects of Rollover Risk: \\ Evidence from Hotels in Crisis}

\author[DeFusco, Nathanson, \& Reher]{
	\textbf{Anthony A. DeFusco\textsuperscript{*}} \quad
	\textbf{Charles G. Nathanson\textsuperscript{\textdagger}} \quad
	\textbf{Michael Reher\textsuperscript{\textdaggerdbl}}
}

\institute{
	\begin{tabular}[h]{c}
		\normalsize \textsuperscript{*}University of Wisconsin--Madison and NBER \\
		\normalsize \textsuperscript{\textdagger}Northwestern University \\
		\normalsize \textsuperscript{\textdaggerdbl} University of California San Diego
	\end{tabular}      
}

\date{}

\begin{frame}[plain, noframenumbering]
	\vfill
	\titlepage
\end{frame}

% *****************************************************************
% presentation slides
% *****************************************************************

\section{Introduction}

\begin{frame}
	\frametitle{Introduction and Motivation}
	
	\visible<1->{
	\textbf{Motivating Questions}
	\vspace{1mm}
	\begin{itemize}
		\setlength{\itemindent}{1em}
		\setlength{\itemsep}{1mm}
		\item How and why do firms adjust operations when debt comes due during a crisis?
		\item Central for understanding how financial frictions amplify recessions
	\end{itemize}
	}
	
	\visible<2->{
	\vspace{3mm}
	\textbf{Prior Evidence}
	\vspace{1mm}
	\begin{itemize}
		\setlength{\itemindent}{1em}
		\setlength{\itemsep}{1mm}
		\item In past crises, firms tended to reduce operations and investment
		\item Primary mechanism: liquidity constraints $\rightarrow$ free up cash to pay off creditors 
		\\ \hspace{1em}% 
		{\color{darkgray}
			\emph{
				\scriptsize[Almeida et al., 2011; Benmelech et al., 2019; Costello, 2020; Granja and Moreira, 2022]
			}
		}
	\end{itemize}
	}
	
	\visible<3->{
	\vspace{3mm}
	\textbf{This Paper}
	\vspace{1mm}
	\begin{itemize}
		\setlength{\itemindent}{0.9em}
		\setlength{\itemsep}{1mm}
		\item New evidence from the most recent crisis
		\item New mechanism: firms \textbf{\textit{strategically}} cut operations to discourage creditor takeover
		\item Model and unique empirical setting to measure these effects
	\end{itemize}
	}
	
\end{frame}

\begin{frame}<1-2>[label=empirical_setting]
	\frametitle{Empirical Setting}
	\begin{itemize}
		\setlength{\itemsep}{3.5mm}
		\item<1-> Quantify the real effects of rollover risk for hotels during COVID pandemic
		\item<2-> Why focus on the hotel industry?
			\vspace{1mm}
			\begin{itemize}
				\setlength{\itemsep}{2mm}
				\item<3-> Hardest hit sector $\rightarrow$ natural point of focus
				\item<4-> Unusually detailed data $\rightarrow$  link building-level outcomes to financial position
				\item<5-> Long-term partially-amortizing balloon loans $\rightarrow$ clean identification
			\end{itemize}
		\item<6-> Empirical approach
			\vspace{1mm}
			\begin{itemize}
				\setlength{\itemsep}{2mm}
				\item Diff-in-diff comparing hotels with loans maturing just before-vs-after COVID
			\end{itemize}
		\item<7-> Ruling in strategic renegotiation as the mechanism
			\vspace{1mm}
			\begin{itemize}
				\setlength{\itemsep}{2mm}
				\item Model to rationalize several results that cannot be explained by liquidity constraints
				\item Additional cross-sectional tests + descriptive facts consistent with the model
			\end{itemize}
	\end{itemize}
\end{frame}

\begin{frame}
	\frametitle{Aggregate Hotel Revenues During the COVID Pandemic}
	\includegraphics[width=0.95\textwidth]{figures/aggregate_revenue_ts.pdf}
\end{frame}

\againframe<3-6>{empirical_setting}

\begin{frame}[label=revenues_raw]
	\frametitle{Revenues for Hotels with Loans Maturing Before-vs-During COVID}
	\includegraphics[width=0.95\textwidth]{figures/revenue_by_maturity_ts.pdf}
\end{frame}

\againframe<6->{empirical_setting}


\begin{frame}[label=results_summary]
\frametitle{Summary of Results}
\begin{itemize}
	\setlength{\itemsep}{2.6mm}
	
	\item<1-> \textbf{Rollover shock has large negative effects on real outcomes}
	\vspace{1mm}
	\begin{itemize}
		\setlength{\itemsep}{1mm}
		\item Sharp \textit{relative} decline in revenue driven by lower occupancy
		\item Accompanying fall in total expenses and \textit{operating profits}
		%\item Robust to non-parametric time trends, e.g. borrower-by-time fixed effects
	\end{itemize}

	\item<2-> \textbf{Results unlikely to be driven by liquidity constraints}
		\vspace{1mm}
		\begin{itemize}
			\setlength{\itemsep}{1mm}
			\item Debt is too large to plausibly pay off by cutting expenses
			\item Short-run operating profits \textit{fall}, leaving less leftover to service debt
			\item Results hold even \textit{within} borrower
			\item Debt is overwhelmingly renegotiated rather than being paid off
		\end{itemize}

	\item<3-> \textbf{Model of strategic renegotiation}
	\vspace{1mm}
	\begin{itemize}
		\setlength{\itemsep}{1mm}
		\item Lenders incur operating adjustment costs when reviving a distressed asset
		\item Borrower can incentivize renegotiation by lowering inputs to make adjustment costly
	\end{itemize}

	\item<4-> \textbf{Variation in real effects is consistent with strategic renegotiation}
	\vspace{1mm}
	\begin{itemize}
		\setlength{\itemsep}{1mm}
		\item V-shaped dynamics of output around renegotiation month
		\item Cross-sectional heterogeneity consistent with adjustment cost mechanism
	\end{itemize}
\end{itemize}
\end{frame}
%
\begin{frame}
\frametitle{Related Literature}
\begin{overprint}

	\onslide<1>
		\begin{itemize}
			\setlength{\itemsep}{3mm}
			\item \textbf{Real effects of corporate debt rollover during crises} \\
			{\scriptsize\color{darkgray} 
				\emph{
					Almeida et al. (2011); 
					Benmelech, Frydman, Papanikolaou (2019); 
					Costello (2020);
					Granja and Moreira (2022)
				}
			}
			\item[] {\small\textbf{Our paper} $\rightarrow$ 
					theory and evidence for a new strategic renegotiation mechanism}
		\end{itemize}

	\onslide<2>
		\begin{itemize}
			\setlength{\itemsep}{3mm}
			\item \textbf{Real effects of corporate debt rollover during crises} \\
			{\scriptsize\color{darkgray} 
				\emph{
					Almeida et al. (2011); 
					Benmelech, Frydman, Papanikolaou (2019); 
					Costello (2020);
					Granja and Moreira (2022)
				}
			}
			\item \textbf{Strategic renegotiation in corporate finance} \\
			{\scriptsize\color{darkgray} 
				\emph{
					Hart and Moore (1994); 
					Benmelech and Bergman (2008)
				}
			}
			\item[] {\small\textbf{Our paper} $\rightarrow$
					explicitly links borrower's operating decisions to negotiating power}
		\end{itemize}
	
	\onslide<3>
		\begin{itemize}
			\setlength{\itemsep}{3mm}
			\item \textbf{Real effects of corporate debt rollover during crises} \\
			{\scriptsize\color{darkgray} 
				\emph{
					Almeida et al. (2011); 
					Benmelech, Frydman, Papanikolaou (2019); 
					Costello (2020);
					Granja and Moreira (2022)
				}
			}
			\item \textbf{Strategic renegotiation in corporate finance} \\
			{\scriptsize\color{darkgray} 
				\emph{
					Hart and Moore (1994); 
					Benmelech and Bergman (2008)
				}
			}			
			
			\item \textbf{Strategic default on CRE debt} \\
			{\scriptsize\color{darkgray} 
				\emph{
					Brown, Ciochetti, Riddiough (2006); 
					Dinc and Y{\"o}nder (2022);
					Glancy, Kurtzman and Lowenstein (2022);
					Glancy et al. (2023);
					Flynn, Ghent, Tchistyi (2024); 
				}
			}
	
			\item[] {\small\textbf{Our paper} $\rightarrow$ 
					jointly studies operating outcomes and loan default}
		\end{itemize}
		
	\onslide<4>
		\begin{itemize}
			\setlength{\itemsep}{3mm}
			\item \textbf{Real effects of corporate debt rollover during crises} \\
			{\scriptsize\color{darkgray} 
				\emph{
					Almeida et al. (2011); 
					Benmelech, Frydman, Papanikolaou (2019); 
					Costello (2020);
					Granja and Moreira (2022)
				}
			}
			\item \textbf{Strategic renegotiation in corporate finance} \\
			{\scriptsize\color{darkgray} 
				\emph{
					Hart and Moore (1994); 
					Benmelech and Bergman (2008)
				}
			}			
			
			\item \textbf{Strategic default on CRE debt} \\
			{\scriptsize\color{darkgray} 
				\emph{
					Brown, Ciochetti, Riddiough (2006); 
					Dinc and Y{\"o}nder (2022);
					Glancy, Kurtzman and Lowenstein (2022);
					Glancy et al. (2023);
					Flynn, Ghent, Tchistyi (2024); 
				}
			}
			\item \textbf{Hotels as an empirical laboratory in corporate finance} \\
			{\scriptsize\color{darkgray} 
				\emph{
					Giroud et al. (2012);
					Kosov{\'a}, Lafontain and Perrigot (2013);  
					Povel et al. (2016); 
					Kosov{\'a} and Sertsios (2018);
					Steiner and Tchistyi (2022);
					Freedman and Kosov{\'a} (2024);
					Spaenjers and Steiner (2024)
				}
			}
		\end{itemize}
			
\end{overprint}
\end{frame}


\begin{frame}
\frametitle{Outline}
\begin{enumerate}
	\setlength{\itemindent}{1em}
	\setlength{\itemsep}{4mm}
	\item Institutional Background and Data
	\item Research Design and Main Results
	\item Model of Strategic Renegotiation
	\item Empirical Evidence for Model Mechanism
	\item Conclusion
\end{enumerate}
\end{frame}

\section{Institutional Background}
\sectionpage

\begin{frame}<2>[label=institutions]
\frametitle<1-6>{Key Facts}
\frametitle<7>{Implications and Open Questions}
\setbeamercovered{transparent}

\begin{overprint}	
	
	\onslide<1>
	
	\begin{enumerate}
		\setlength{\itemindent}{1em}
		\setlength{\itemsep}{3mm}
		\item Hotels finance themselves with long-term debt paid at maturity
		\vspace{2mm}

		\item Hotels' long-term debt is large and concentrated in a single loan
		\vspace{2mm}

		\item COVID crisis sharply reduced hotel demand for a potentially long period
		\vspace{2mm}
		
		\item Debt maturing during the crisis was renegotiated
		\vspace{2mm}		
		
		\item Hotel owners can exert significant control over operations in a crisis		

	\end{enumerate}
	
	\onslide<2>
	
	\begin{enumerate}
		\setlength{\itemindent}{1em}
		\setlength{\itemsep}{3mm}
		\item Hotels finance themselves with long-term debt paid at maturity
			\item[] {\small  $\rightarrow$ 
					maturity date is a debt rollover event}
		\vspace{2mm}
		\item<3> Hotels' long-term debt is large and concentrated in a single loan
			
		\vspace{2mm}
		\item<4> COVID crisis sharply reduced hotel demand for a potentially long period
			
		\vspace{2mm}
		\item<5> Debt maturing during the crisis was renegotiated
			
		\vspace{2mm}
		\item<6> Hotel owners can exert significant control over operations in a crisis
			
	\end{enumerate}

	\onslide<3>
	
	\begin{enumerate}
		\setlength{\itemindent}{1em}
		\setlength{\itemsep}{3mm}
		\item Hotels finance themselves with long-term debt paid at maturity

		\vspace{2mm}
		\item Hotels' long-term debt is large and concentrated in a single loan
			\item[] {\small  $\rightarrow$ 
					debt rollover events are substantial}
			
		\vspace{2mm}
		\item<4> COVID crisis sharply reduced hotel demand for a potentially long period
			
		\vspace{2mm}
		\item<5> Debt maturing during the crisis was renegotiated
			
		\vspace{2mm}
		\item<6> Hotel owners can exert significant control over operations in a crisis
			
	\end{enumerate}

	\onslide<4>
	
	\begin{enumerate}
		\setlength{\itemindent}{1em}
		\setlength{\itemsep}{3mm}
		\item Hotels finance themselves with long-term debt paid at maturity

		\vspace{2mm}
		\item Hotels' long-term debt is large and concentrated in a single loan
			
		\vspace{2mm}
		\item COVID crisis sharply reduced hotel demand for a potentially long period

			\item[] {\small  ... PWC (2020): demand will take as long to recover as Great Recession}

			\item[] {\small  $\rightarrow$ 
					uncertainty over duration of the shock creates incentives to renegotiate debt}
			
		\vspace{2mm}
		\item<5> Debt maturing during the crisis was renegotiated
			
		\vspace{2mm}
		\item<6> Hotel owners can exert significant control over operations in a crisis
			
	\end{enumerate}
	
	\onslide<5>
	\begin{enumerate}
		\setlength{\itemindent}{1em}
		\setlength{\itemsep}{3mm}
		\item Hotels finance themselves with long-term debt paid at maturity

		\vspace{2mm}
		\item Hotels' long-term debt is large and concentrated in a single loan
			
		\vspace{2mm}
		\item COVID crisis sharply reduced hotel demand for a potentially long period
				
		\vspace{2mm}
		\item Debt maturing during the crisis was renegotiated

			\item[] {\small  ... nearly all loan modifications are term extensions}

			\item[] {\small  ... no government-mandated forbearance in CRE}

			\item[] {\small  $\rightarrow$ 
					private incentives of lenders to offer term extensions matter}
					
		\vspace{2mm}
		\item<6> Hotel owners can exert significant control over operations in a crisis
			
	\end{enumerate}			

	\onslide<6>
	
	\begin{enumerate}
		\setlength{\itemindent}{1em}
		\setlength{\itemsep}{3mm}
		\item Hotels finance themselves with long-term debt paid at maturity

		\vspace{2mm}
		\item Long-term debt is large and concentrated in a single loan
			
		\vspace{2mm}
		\item COVID crisis reduced hotel demand for a potentially long period
				
		\vspace{2mm}
		\item Debt maturing during the crisis was renegotiated
		
		\vspace{2mm}
		\item Hotel owners can exert significant control over operations in a crisis
			\item[] {\small  $\rightarrow$ 
					owners' debt position may matter for operating outcomes}

	\end{enumerate}
	
	\onslide<7>
	\vspace{2mm}
	\begin{itemize}
		\setlength{\itemsep}{3mm}
		\item Hotel owners with debt maturing in crisis have face a significant rollover shock \\
		{\color{darkgray}\emph{\small [Facts \#1 and \#3]}}
		\vspace{1mm}	
		\item They can potentially modify operations to respond to this shock... \\
		{\color{darkgray}\emph{\small[Fact \#5]}}
		\item But why would they?
		\vspace{2mm}
			\begin{itemize}
				\setlength{\itemsep}{1mm}
				\item Cut costs to pay off the debt?
				\item[--] Implausible this would generate enough cash \\
				{\color{darkgray}\emph{\small[Fact \#2]}}	
				\item Lose incentive to maintain hotel because creditors will seize it?
				\item[--] Inconsistent with widespread renegotiation \\
				{\color{darkgray}\emph{\small[Fact \#4]}}
			\end{itemize}
		\item \textbf{Open Questions:} Do hotel operations respond to rollover shock and why?			
	\end{itemize}		
	
\end{overprint}

\end{frame}

\begin{frame}
\frametitle{Prepayment Penalties and Principal Balance Remaining at Maturity}
\vspace{-2mm}
\begin{center}
\begin{tabular}{cc}
	\large \textbf{Loans in Free Prepayment} & 
	\large \textbf{Principal Balance Remaining} \\
	\includegraphics[width= 0.48\textwidth]{figures/share_in_free_prepayment}&
	\includegraphics[width= 0.48\textwidth]{figures/share_balance_outstanding}\\
	\multicolumn{2}{l}{\tiny Sample: CMBS loans with 10-year maturity \& maturity date $\geq12$ months pre-COVID}
\end{tabular}
\end{center}	
\end{frame}

\againframe<3>{institutions}

\begin{frame}
\frametitle{Operating Profits Available to Service Debt}
\begin{center}
\includegraphics[width=0.70\textwidth]{figures/ebitda_to_balloon_histogram.pdf}
\end{center}
\end{frame}

\againframe<4>{institutions}

\begin{frame}
\frametitle{Aggregate Hotel Revenue}
\includegraphics[width=0.95\textwidth]{figures/aggregate_revenue_ts.pdf}
\end{frame}

\againframe<5>{institutions}

\begin{frame}
\frametitle{Resolution of Loans Scheduled to Mature in Early Pandemic}
\includegraphics[width=0.95\textwidth]{figures/resolution.pdf}
\end{frame}

\againframe<6>{institutions}

\begin{frame}
\frametitle{Hotel Ownership and Operations}
\begin{itemize}
	\setlength{\itemsep}{4mm}
	\item<1-> Most hotels are branded but owned by separate entity
	\vspace{1mm}
	\begin{itemize}
		\setlength{\itemsep}{1.5mm}
		\item Marriott vs. Host Hotels
		\item Owners range from mom-and-pops to exchange-traded REITs
	\end{itemize}
	
	\item<1-> Operate under one of three arrangements
	\\ {\color{darkgray}\emph{\scriptsize Freedman \& Kosov\'{a} (2014); Kosov\'{a} \& Sertsios (2018)}}
	\vspace{1mm}
	\begin{itemize}
		\setlength{\itemsep}{1.5mm}
		\item Owner-operated
		\item Brand-managed
		\item Third-party non-brand managed
	\end{itemize}
	
	\item<1-> In the case of delegated management, owners can exert control in a crisis
	
	\vspace{1mm}
	\begin{itemize}
		\setlength{\itemsep}{1.5mm}
		\item Direct: Can withhold operating capital required to meet cash shortfalls \\
		{\color{darkgray}\emph{\scriptsize (e.g. Sunstone Hotel Properties (2004))}}
		\item Effective: Subordination clauses $\rightarrow$ incentives vis-{\a`a}-vis creditors are aligned \\ 
		{\color{darkgray}\emph{\scriptsize (e.g. Marriott 10-K (2020))}}
	\end{itemize}
\end{itemize}
\end{frame}

\againframe<1,7>{institutions}

\section{Data}
\sectionpage

\begin{frame}
\frametitle{Data Sources and Sample Selection}

\begin{itemize}
	\setlength{\itemsep}{3mm}
	\item<1-> \textbf{STR hotel performance data}
	\vspace{1mm}
	\begin{itemize}
		\item Anonymized panel of hotels
		\item Static characteristics for each hotel
		\item Daily data $\rightarrow$ room revenue, occupancy, price 
		\item Annual data $\rightarrow$ category-level revenue and expenses, earnings
	\end{itemize}
	
	\item<1-> \textbf{Trepp CMBS loan servicing data}
	\vspace{1mm}
	\begin{itemize}
		\item Contract terms
		\item Origination characteristics
		\item History of payments, modifications, and disposition
		\item Supplement with CLTV and borrower RE assets from Real Capital Analytics (RCA)
		\item Property address $\rightarrow$ permits match to STR and RCA
	\end{itemize}
	
	\item<1-> \textbf{Primary sample restrictions}
	\vspace{1mm}
	\begin{itemize}
		\item STR hotels with CMBS loan in Trepp maturing 2018-2022
		\item Pandemic maturity [``treatment''] $=$ scheduled to mature in Feb 2020--Jan 2021 
		\item Pre-pandemic maturity [``control''] $=$ scheduled to mature in Feb 2019--Jan 2020
	\end{itemize}
\end{itemize}
\end{frame}

\begin{frame}
\frametitle{Descriptive Statistics}
\vspace{-8mm}	
\begin{center}
	\resizebox{0.725\textwidth}{!}{		
		\begin{threeparttable}
			\estauto{tables/sum_stats}{4}{S[table-column-width=0.75in, table-format=1.2]}
			%\estauto{tables/sum_stats_short}{4}{S[table-column-width=0.75in, table-format=1.2]}
		\end{threeparttable}
	}
\end{center}
\end{frame}

\begin{frame}[label=research_design]
\frametitle{Research Design: Difference in Differences}
\begin{itemize}
	\setlength{\itemsep}{5mm}
	\item Compare hotels with early/late maturing loans, pre/post COVID
	\setlength{\belowdisplayskip}{0pt}		
	\begin{equation*}
	y_{imt} = 
	\alpha_i + 
	\delta_{mt} + 
	\gamma X^{\prime}_{it} + 
	\sum_{\tau = \underline{t}}^{\tau= \overline{t}}
	\Big[
	\alert{
		\beta_\tau \times 
		Pandemic Maturity_i \times 
		\mathbbm{1}_{t = \tau}
	}
	\Big] + 
	\epsilon_{it}
	\end{equation*}	
	\item Identifying assumption: parallel trends for hotels with early-vs-late maturing loans
	\item Key threat: differential demand sensitivity for early maturing loans
	\vspace{1.5mm}
	\begin{itemize}
		\setlength{\itemsep}{1.5mm}
		\item Interact hotel X's with month FEs $\rightarrow$ accommodates many differential trends
		\item Most extreme: Chain-Market-Month FEs, Borrower FEs
	\end{itemize}
\end{itemize}
\end{frame}

\section{Real Effects of Debt Rollover}
\sectionpage

\begin{frame}[label=revenue]
\frametitle{Effect of Pandemic Maturity: Log(Room Revenue)}
\includegraphics[width=0.95\textwidth]{figures/es_revenue_baseline.pdf}

\begin{textblock*}{3cm}(-0.04\textwidth, -0.02\textheight)
	\hyperlink{closure}{\beamerbutton{Closure}}     
\end{textblock*}

\end{frame}

\begin{frame}
\frametitle{Decomposition: Occupancy-vs-Prices}
\includegraphics[width=0.95\textwidth]{figures/es_occ_adr.pdf}
\end{frame}

\begin{frame}
\frametitle{Controlling for Differential Trends}
\resizebox{\textwidth}{!}{		
	\begin{threeparttable}
		\estauto{tables/dd_revenue}{6}
		{S[table-column-width=0.75in, table-format=1.3]}
	\end{threeparttable}
}	
\begin{textblock*}{6cm}(-0.04\textwidth, 0.14\textheight)
	\hyperlink{dd_chain_borrower_fe}{\beamerbutton{Chain-by-Market-by-Month FE}}     
\end{textblock*}

\end{frame}

\begin{frame}
\frametitle{Alternative Maturity Bandwidths}

\resizebox{\textwidth}{!}{		
	\begin{threeparttable}
		\estauto{tables/dd_revenue_bandwidths}{4}
		{S[table-column-width=0.75in, table-format=1.3]}
	\end{threeparttable}
}
\end{frame}

\begin{frame}\label{expenses}
\frametitle{Effects on Hotel Inputs}

\only<1>{	
	\begin{center}
		\begin{tabular}{cc}
			\large \textbf{Log(Total Revenue)} & 
			\large \textbf{Log(Total Expenses)} \\
			\hspace{-5mm} \includegraphics[width=0.5\textwidth]
			{figures/es_revenue_yearly.pdf} &
			\hspace{-5mm}\includegraphics[width= 0.5\textwidth]
			{figures/es_expense_yearly.pdf} 
		\end{tabular}
	\end{center}
}

\only<2>{	
	\begin{center}
		\begin{tabular}{cc}
			\large \textbf{Log(Labor Expenses)} & 
			\large \textbf{Log(Sales \& Marketing Expenses)} \\
			\hspace{-5mm} \includegraphics[width=0.5\textwidth]
			{figures/es_labor_yearly.pdf} &
			\hspace{-5mm}\includegraphics[width= 0.5\textwidth]
			{figures/es_sm_yearly.pdf} 
		\end{tabular}
	\end{center}
	\begin{textblock*}{4cm}(-0.04\textwidth, 0.06\textheight)
	\hyperlink{marketing}{\beamerbutton{Marketing Expense Monthly}}     
	\end{textblock*}
}
\end{frame}

\begin{frame}
\frametitle{Effect on Hotel Operating Profits}
\begin{center}
\includegraphics[width=0.72\textwidth]{figures/es_profit_yearly.pdf}
\end{center}
\end{frame}

\againframe<1-2>{results_summary}

\section{Model}
\sectionpage


\begin{frame}
\frametitle{Production Environment}

\begin{itemize}
	\setlength{\itemsep}{6mm}
	
	\item Production function: $	F(K,L_t) = K^{1-\alpha}L_t^\alpha$ 
	
	\item Discrete time
	\vspace{1mm}
	\begin{itemize}
	\setlength{\itemsep}{2mm}
	\item Economy begins in ``high state" at $t = 0$
	\item Crisis at $t = 1$ reduces price of output to ``low state" value
	\item With probability $q$ crisis ends at $t = 2$ and returns to ``high state" forever
	\end{itemize}

	\item Owners can costlessly adjust, but lenders incur adjustment costs:

\begin{equation*}
			\phi(L_t,L_{t-1}) = 
			\begin{cases}
				0, & L_t \leq L_{t-1}\\
				\frac{\gamma}{2}\left(\frac{L_t}{L_{t-1}}-1\right)^2 L_{t-1}, & L_t > L_{t-1},
			\end{cases}
		\end{equation*}

	\item Cost parameter $\gamma \sim G(\gamma)$ known to lenders, but not to borrowers 

\end{itemize}
\end{frame}


\begin{frame}
\frametitle{Debt}
\begin{itemize}
	\setlength{\itemsep}{4mm}
	\item Terms of debt are exogenous
	\vspace{1mm}
	\begin{itemize}
		\setlength{\itemsep}{2mm}
		\item Principal = $\tilde{D}$, debt service = $r\tilde{D}$
		\item No amortization $\rightarrow$ balloon payment at maturity: $D = (1+r)\tilde{D}$
	\end{itemize}
	\item Two types of borrowers:
	\vspace{1mm}
	\begin{itemize}
		\setlength{\itemsep}{2mm}
		\item \textit{Crisis maturity borrowers}: debt due at time 1 (crisis onset)
		\item \textit{Non-crisis maturity borrowers}: debt due at time 2 (uncertainty resolution date)
	\end{itemize}
	\item Timing of events at maturity:
	\vspace{1mm}
	\begin{itemize}
		\setlength{\itemsep}{2mm}
		\item Borrower chooses labor for that period
		\item Borrower makes required loan payment or defaults
		\item If default, lender forecloses OR at time 1 offers to extend balloon maturity to time 2
		\item If extension offered, borrower accepts or rejects \& gives property to lender
	\end{itemize} 	
\end{itemize}
\end{frame}


\begin{frame}
\frametitle{Strategic Renegotiation}
\begin{proposition}
The probability of receiving forbearance conditional on defaulting continuously and weakly decreases in the borrower's choice of labor at time 1, $L_1$, between 0 and the static optimum, $L^*(p_l)$. The probability limits to 1 for small values of labor and is less than 1 at the static optimum if $D < D^{**}$
\end{proposition} 

\vspace{2mm}
\textbf{Intuition}
\vspace{1mm}
\begin{itemize}
	\setlength{\itemindent}{1em}
	\setlength{\itemsep}{1mm}
	\item Lower inputs at time 1 $\rightarrow$ lower time-1 profits \& higher time-2 adjustment costs
	\item Higher time-2 adjustment costs $\downarrow$ value of foreclosure in any state at time 2
	\item Value of foreclosure $\downarrow$ more than of forbearance, so lender more likely to forbear
\end{itemize}

\end{frame}

\begin{frame}<2->
\frametitle{Strategic Default and Real Effects of Debt Rollover}

\vspace{-5mm}
\begin{overprint}

	\onslide<1>
	\begin{proposition}
			\begin{itemize}
				\setlength{\itemsep}{4mm}
				\item If $D<D^{*}$, both borrowers pay debt and set $L_1 = L^*(p^l)$.
				\item If $D\in(D^*,D^{**})$, the crisis-mat. borrower defaults, sets $L_1 < L^*(p^l)$, and 
				receives forbearance with positive prob.; non-crisis mat. pays debt and sets $L_1 = 
				L^*(p^l)$.
				\item If $D>D^{**}$, both borrowers default, lender forecloses and determines $L_1$
				\item The default thresholds satisfy:
				\begin{equation*}
					V_{low} < D^* < (1-q)V_{low} + qV_{high} < D^{**},
				\end{equation*}
				were $V_s = $ debt free value of the firm if state $s$ realized at time 2.  
			\end{itemize}
		\end{proposition}
		
	\onslide<2>
	\begin{proposition}
		\begin{itemize}
			\setlength{\itemsep}{4mm}
			\item If $D<D^{*}$, both borrowers pay debt and set $L_1 = L^*(p^l)$.
			\item If $D\in(D^*,D^{**})$, the crisis-mat. borrower defaults, sets $L_1 < L^*(p^l)$, and 
			receives forbearance with positive prob.; non-crisis mat. pays debt and sets $L_1 = 
			L^*(p^l)$.
			\item If $D>D^{**}$, both borrowers default, lender forecloses and determines $L_1$
		\end{itemize}
	\end{proposition}
		
	\vspace{2mm}
	\begin{cor}
	 	Revenue, output, labor, and profits are weakly lower for crisis maturity firms at time 1 than 
	 	for non-crisis maturity firms; the relation is strict if $D\in(D^*,D^{**})$.
	\end{cor} 

\end{overprint}

\end{frame}

%\begin{frame}
%\frametitle{Strategic Default and Real Effects of Debt Rollover}
%\vspace{2mm}
%\textbf{Intuition for Default Decision}
%\vspace{1mm}
%\begin{itemize}
%	\setlength{\itemindent}{1em}
%	\setlength{\itemsep}{1mm}
%	\item Defaulting at time 1 risks foreclosure at time 1
%	\item[]... crisis-maturity borrowers with $D > D^{*}$ are willing to risk foreclosure
%	\item[]... those with $D < D^{*}$ are not because they have more equity if crisis ends
%\end{itemize}
%\vspace{2mm}
%\textbf{Implications for Real Activity}
%\vspace{1mm}
%\begin{itemize}
%	\setlength{\itemindent}{1em}
%	\setlength{\itemsep}{1mm}
%	\item Borrowers not in default set inputs at static optimum at time 1
%	\item Lowering inputs below this static optimum causes a \textit{second-order} loss in profits 
%	\item ... but also causes a \textit{first-order} gain in probability of forbearance 
%	%\item So early-maturity borrowers with $D > D^{**}$ lower inputs below the optimum
%	\item Lower inputs $\rightarrow$ larger $\downarrow$ in revenue, output, and profits
%\end{itemize}
%\end{frame}

\begin{frame}
\frametitle{Adjustment Costs and the Real Effects of Debt Rollover}

\vspace{-5mm}

	\begin{proposition}
		If the distribution of $\gamma$ shifts toward $0$ from $G(\gamma)$ to $G_a(\gamma) = G(a\gamma)$, for $a > 1$ 
		then:
		\vspace{1mm}
		\begin{itemize}
			\setlength{\itemsep}{4mm}
			%\setlength{\itemindent}{1em}
			\item The debt range where strategic default occurs shrinks: 
				$D_a^* \geq D^*$ and $D_a^{**} = D^{**}$
			\item If $log(1-G)$ is concave, then conditional on strategic default operations at the 
			firm fall: $L_{1,a}^* \leq L_1^*$ when $D \in (D_a^*,D_a^{**})$, with strict inequality 
			when $L_1^* > 0$.
		\end{itemize}
	\end{proposition}
		
	
	\vspace{2mm}
	\textbf{Intuition}
	\vspace{1mm}
	\begin{itemize}
		\setlength{\itemindent}{1em}
		\setlength{\itemsep}{1mm}
		\item Borrowers are less likely to default when lenders cost of foreclosure is low, but...
		\item Conditional on defaulting, they must cut inputs by more to ward off foreclosure 
		\item Empirically, most loans do not pay off on time $\rightarrow$ the second effect dominates?
	\end{itemize}


\end{frame}

\section{Empirical Evidence for Model Mechanism}
\sectionpage

\begin{frame}<1-2>[label = predictions]
	\frametitle{Empirical Predictions of the Model}
	
	\begin{enumerate}
		\setlength{\itemsep}{4mm}
		\item \textbf<1>{Lower output, inputs, and profits for hotels facing a crisis rollover}
		\item \textbf<2>{Reason for lower real outcomes = renegotiation $\rightarrow$ rebound after modification?}
		\item\textbf<3>{Effects concentrated at higher debt levels}
		\item \textbf<4>{No effect for loans with built-in extension option}
		\item \textbf<5>{Smaller effect for loans with cash sweep/lockbox feature}
		\item \textbf<6>{Larger real effects when facing lenders with lower adjustment costs}
	\end{enumerate}
			
\end{frame}


\begin{frame}[label = modification]
\frametitle{Average Revenue around Modification Month for Treated Hotels}
\begin{center}
\includegraphics[width=0.85\textwidth]{figures/dynamics_revenue_around_modification.pdf}
		\vspace{-7mm}
\end{center}
\begin{textblock*}{3cm}(-0.04\textwidth, 0.05\textheight)
	\hyperlink{es_modification}{\beamerbutton{Event Study}}     
\end{textblock*}
\end{frame}

\againframe<3>{predictions}


\begin{frame}
\frametitle{Heterogeneity in Revenue Estimates by LTV}
\includegraphics[width=0.95\textwidth]{figures/es_revenue_het_ltv_f_67.pdf}
\end{frame}

\begin{frame}<1>[label = het_table]
\frametitle{Heterogeneity in Effect on Hotel Revenues}

	\begin{table}[H]
	\resizebox{\textwidth}{!}{		
		\begin{threeparttable}
			\estauto{tables/dd_mechanisms}{6}{S[table-column-width=0.65in, table-format=1.3]}
		\end{threeparttable}
	}
	\end{table}

	\tikzset{external/export next=false}
	\begin{tikzpicture}[overlay]
		%\draw[lightgray,very thin] (0,-8) grid (24,8);
		\draw<1>[orangered,ultra thick] (6,6.1) rectangle (7.4,7.5);
		\draw<2>[orangered,ultra thick] (7.5,5.5) rectangle (8.8,7.5);
		\draw<3>[orangered,ultra thick] (8.9,4.8) rectangle (10.2,7.5);
		\draw<4>[orangered,ultra thick] (10.3,2.9) rectangle (14.4,7.5);
	\end{tikzpicture}
	
\end{frame}

\againframe<4>{predictions}

\againframe<2>{het_table}

\againframe<5>{predictions}

\againframe<3>{het_table}

\againframe<6>{predictions}

\begin{frame}
	\frametitle{Proxying for Lender's Adjustment Costs}

	\textbf{Hotels per Chain}
	\vspace{1mm}
	\begin{itemize}
		\setlength{\itemindent}{1em}
		\item Proxy for how ``common'' a hotel chain is
		\item Idea: Easier to operate a more ubiquitous hotel than a specialty one	
	\end{itemize}		

	\vspace{2mm}
	\textbf{Chains per Servicer}
	\vspace{1mm}
	\begin{itemize}
		\setlength{\itemindent}{1em}
		\item Proxy for lender's experience operating differing types of hotels
		\item Idea: Servicers with diverse experiences can more easily operate a given hotel	
	\end{itemize}
	
	\vspace{2mm}
	\textbf{Servicer Chain Share}
	\vspace{1mm}
	\begin{itemize}
		\setlength{\itemindent}{1em}
		\item Proxy for lender's experience operating a particular type of hotel
		\item Idea: Servicers with deep experience in a given chain can easily operate that chain
	\end{itemize}		
	
\end{frame}

\againframe<4>{het_table}


\againframe<2,4>{results_summary}

\section{Conclusion}

\begin{frame}
\frametitle{Conclusion}

\textbf{Corporate Finance}
\vspace{1mm}
\begin{itemize}
	\setlength{\itemindent}{1em}
	\item Renegotiation impairs real outcomes when foreclosure costs are endogenous
	\item Channel seems relevant in our context; potentially in others (but harder to isolate)
\end{itemize}		

\vspace{4mm}
\textbf{CRE Mortgage Design}
\vspace{1mm}
\begin{itemize}
	\setlength{\itemindent}{1em}
	\item Balloon mortgages can have real costs by altering incentives at maturity
	\item Macro-prudential regulation of CRE: LTV vs staggering of debt maturities
%	\vspace{1mm}			
%	\begin{itemize}
%		\setlength{\itemindent}{0.5em}
%		\setlength{\itemsep}{1mm}
%		\normalsize
%		\item This paper: hotels during COVID
%		\item Liebersohn, Correa \& Sicilian (2022): malls during rise of e-commerce
%		\item Gupta, Mittal \& Van Nieuwerburgh (2023) : offices due to work from home?
%	\end{itemize}						
	
\end{itemize}	
\end{frame}

%\begin{frame}
%\begin{center}
%\includegraphics[scale=0.5]{figures/wsj}
%\vskip5mm
%\includegraphics[scale=0.7]{figures/wsj_title}
%\end{center}
%\end{frame}

\section{Thanks!}
\sectionpage

\section{Linked Slides}
\sectionpage

\setbeamertemplate{footline}{}

\begin{frame}[noframenumbering]\label{closure}
\frametitle<1>{Effect of Pandemic Maturity: Closure}

\includegraphics[width=0.95\textwidth]{figures/es_closed.pdf}

\begin{textblock*}{3cm}(0.98\textwidth, -0.015\textheight)
	\hyperlink{revenue}{\beamerbutton{<Back}}     
\end{textblock*}	
\end{frame}

\begin{frame}[noframenumbering]\label{dd_chain_borrower_fe}
\frametitle<1>{Chain-by-Market and Borrower-by-Month FE analysis}

\resizebox{\textwidth}{!}{		
	\begin{threeparttable}
		\estauto{tables/dd_chain_msa_fe}{6}
		{S[table-column-width=0.75in, table-format=1.3]}
	\end{threeparttable}
}

\begin{textblock*}{3cm}(0.98\textwidth, 0.11\textheight)
	\hyperlink{robustness_trends}{\beamerbutton{<Back}}     
\end{textblock*}

\end{frame}

\begin{frame}[noframenumbering]\label{marketing}
\frametitle<1>{Monthly Marketing Expense Analysis}

\includegraphics[width=0.95\textwidth]{figures/time_by_post_reg_sm_expense_month_pl}

\begin{textblock*}{3cm}(0.98\textwidth, -0.02\textheight)
	\hyperlink{expenses}{\beamerbutton{<Back}}     
\end{textblock*}

\end{frame}

\begin{frame}[label = es_modification]
\frametitle{Effect on Revenue around Modification Month}
\includegraphics[width=0.95\textwidth]{figures/es_revenue_around_modification.pdf}
\begin{textblock*}{3cm}(0.98\textwidth, -0.02\textheight)
	\hyperlink{modification}{\beamerbutton{<Back}}     
\end{textblock*}
\end{frame}




\end{document}
